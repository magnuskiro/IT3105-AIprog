\documentclass[titlepage]{article}
\usepackage{amsmath}
\usepackage[utf8]{inputenc}

\author{Jan Alexander Bremnes\\Magnus Kirø}
\title{IT3105 - Ex 1\\Texas Hold'em}
\date{September 2011}

\begin{document}

    \maketitle

    \tableofcontents

\section*{TODO}
\begin{itemize}
\item   Integrate modules for hand strength calculation, so that the players can use the information to influence betting behaviour
\item   Change getAction in strategy to use the pre flop rollout-table
\item   CODE ERROR!! in playpoker line #178 change to remainingplayers
\item   Do 5 test runs of 1000 games to produce statistics and update this report
\item   Start looking at opponent modelling!
\section*{Introduction}
This is all about craeting Texas Hold'em bots and making them smart.

\section*{Point distribution and teask completion}
\begin{itemize}
\item A well-functioning Phase I player that obeys all fundamental poker rules. (20 points)

\item Five different 1000-hand runs of a game involving 4 or more Phase-I players, with the results of each 1000-game segment summarized in a table showing the final resource (e.g. money) of each player. (5 points)

\item Modules for computing, storing (to file) and loading (for use in a poker game) the pre-flop rollout probabilities for k = 2 to 10 player games. Each rollout for each card equivalence class should involve at least 1000 rounds of simulated play. (20 points)

\item Modules for estimating hand strength for a pair of hole cards plus 3, 4 or 5 community/shared cards. This will involve testing the given hand against hands made from all possible pairs of hole cards (using the remaining 47 to 45 cards), as described in ai-poker-players.pdf and diagramed in Figure 15 of that document. (15 points)

\item Five different 1000-hand runs of a game involving 4 or more Phase-II players, with the results of each 1000-game segment summarized in a table, as above. (10 points)

\item Modules for performing basic opponent modeling. (15 points)

\item Five different 1000-hand runs of 4 or more players, at least of 2 of which use Phase-III techniques, while the others may use only Phase-II methods. Include the standard summary table of each 1000-game segment. (10 points)

\item The general quality of the report, which must cover all of the 5 topics mentioned above. (5 points).
\end{itemize}

\section*{Report Requirements}
\begin{itemize}
\item The basic structure of your code for each project phase,

\item The logic behind the betting decisions made by the players in each of the project phases that you complete,

\item The opponent models used in phase III, in particular, the contexts used to differentiate game situations,

\item The results of multi-hand (1000) runs for each of the project phases that you complete. These should not be screen dumps, but simple tables indicating each players total winnings. Five simple tables per project phase is all that is required (and desired) here.

\item A brief discussion of why (you believe) certain strategies were more (or less) effective than others for
each phase of the project
\end{itemize}

\pagenumbering{arabic}
\newpage

\section{Introduction}
	This report describes our work done in exercise 1 of IT1305, which were to program a poker bot. 
	
\section{Code Structure}
	\subsection{Phase 1}
		new-game.py starts the program, and let's the user choose between playing poker, do a test run, or debug. Debugging runs the program with the chosen number of players and rounds, and the game information is printed to screen. As we have not implemented support for human vs computer players, the choice between debugging and play game is a false one, both of them do the same thing. The only difference between play-poker.py and run-test.py, is that play-poker prints gameplay information, while run-test.py only prints the final results after having gone through the chosen number of rounds. Each game creates a list of player-objects, which are given a strategy, a player number, to keep track of who's who, and an initial amount of money. A table-object keeps track of the community cards, the amount of money in the pot, and the current bet. The game-state-object stores the players and table objects to allow easier passing of information between the different parts of the program. Every player has a strategy object, which it uses to decide on betting actions. The file strategy.py can be edited to change the existing strategies or add more. When a player is required to perform an action, the method evaluate-hand, in the module betting.py. is called and depending on the strength of the hand, and the players strategy, the player either folds, calls/checks or raises. The players uses the cards.py module to calculate the strength of their hands. In the pre-flop betting round, these are the only modules that are used. The moduels card.py is the helper code which was supplied to us, with a minor change added so that it's possible to remove cards from a card deck.
		
	\subsection{Phase 2}
		For this phase of the project, we added readPreFlopRollouts.py to let the players decide which equivalence class their hand belongs to, and find the probability of it winning in the rollout-table. To calculate the probabilities after the flop, handstrength.py is used. This module calculates the probability of the players hand, when supplied with the hole cards, the community cards and the number of remaining players. It uses the formula described in ai-poker-players.pdf. The rollout-table is generated with the help of roll-out*.py where * = [1-3]. These modules simulate thousands of games for every possible combination of hole cards, and we use three different modules because we thought it easier to just let the three run simultaneously on a multi core computer, rather than rewriting the code to use threads. The resulting file is then read by scanner.py, which counts the numbers of wins, losses and draws for every equivalence class and calculates the winning probabilities, which are stored in results.txt

\section{Where we explain the logic behind the betting decisions}
	The poker players have different strategies for betting in each of the project phases. They differ from phase to phase, as the amount of information available to the players increase
	
	\subsection{Phase 1 - Primitive strategies}
		WARNING! INFORMATION MAY NOT BE CORRECT! CONSULT THE STRATEGY.PY TO SEE HOW THE 	STRATEGIES WORK
In this phase, as with the others, we have included three different player strategies. A 		conservative player, which usually folds when he has a weak hand, often calls on mediocre hands, 	but only raises on strong hands. Then there is the normal players, which does a little bit of 		everything. Sometimes it folds on mediocre hands, sometimes it raises on weak hands, and 		sometimes it only calls on strong hands. The aggressive player sometimes raise on weak hands, but always strong and often on mediocre. In phase 1 of the project, the player strategies are crude and random for the most part. Betting 	in the pre flop round, is based solely on randomness, as there is no information for the players to base their strategies on. 
	
	\subsection{Phase 2 - Strategies based on pre-flop roll-outs and hand strengths}
		In this phase, we generated the pre-flop roll-out table and added a hand strength calculator to let the players make a better guess at how good a chance they have of winning. The roll-out table is used prior to the flop, and is calculated from a simulation of 50,000 games for all numbers of opponents, and all possible hands. The players look up their hand in the table, and use the probability of their hand winning the game, to influence their betting. Post-flop this table is of no more use, as it only based on the possibility of a given two card combination winning the game, so the hand strength calculator is used instead. 
		
	\subsection{Phase 3 - Strategies based on opponent modelling}
		NOT YET IMPLEMENTED
		
\section{Future work}
	Players strategies can be improved by letting them take more information about the game state into account. The current strategies only use the information about the players themselves. They do not take into consideration all the other information that is present, such as number of remaining players, how big the current pot is and how hight the current bet is. If a player has a mediocre hand, there's only two remaining opponents out of the original seven, and the current bet is a small fraction of the players current money, then it would make more sense to place a bet, or raise, then it would if there were seven opponents and a bet a quarter of the players money.

\section{Player winnings tables}
	\subsection{Phase-1 player winnings table}
		5-players, 1000-games, 100-start money \\
		\begin{matrix}
		  	\\
			game set:    &     1  &     2  &     3 &     4 &     5 \\
			Player 0 & -4600  &   850  & -2290 &  1970 &  3730 \\
			Player 1 & -3430  &  2590  &  1690 &   280 &   580 \\
			Player 2 &  1520  &  2980  &  1100 &  -390 &  4390 \\
			Player 3 &  -120  & -1300  &  -350 & -1930 & -5390 \\
			Player 4 &  7130  & -4620  &   350 &   570 & -2810 \\
		\end{matrix}

	\subsection{Phase-2 player winnings table}
		5-players, 1000-games, 100-start money \\
		\begin{matrix}
			\\
			game set:    &     1  &     2  &     3 &     4 &     5 \\
			Player 0 & -4600  &   850  & -2290 &  1970 &  3730 \\
			Player 1 & -3430  &  2590  &  1690 &   280 &   580 \\
			Player 2 &  1520  &  2980  &  1100 &  -390 &  4390 \\
			Player 3 &  -120  & -1300  &  -350 & -1930 & -5390 \\
			Player 4 &  7130  & -4620  &   350 &   570 & -2810 \\
		\end{matrix}
		
	\subsection{Phase-3 player winnings table}
		5-players, 1000-games, 100-start money \\
		\begin{matrix}
			\\
			game set:    &     1  &     2  &     3 &     4 &     5 \\
			Player 0 & -4600  &   850  & -2290 &  1970 &  3730 \\
			Player 1 & -3430  &  2590  &  1690 &   280 &   580 \\
			Player 2 &  1520  &  2980  &  1100 &  -390 &  4390 \\
			Player 3 &  -120  & -1300  &  -350 & -1930 & -5390 \\
			Player 4 &  7130  & -4620  &   350 &   570 & -2810 \\
		\end{matrix}



\end{document}
